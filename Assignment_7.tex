\documentclass[journal,12pt,twocolumn]{IEEEtran}

\usepackage{setspace}
\usepackage{gensymb}

\singlespacing


\usepackage[cmex10]{amsmath}

\usepackage{amsthm}

\usepackage{mathrsfs}
\usepackage{txfonts}
\usepackage{stfloats}
\usepackage{bm}
\usepackage{cite}
\usepackage{cases}
\usepackage{subfig}

\usepackage{longtable}
\usepackage{multirow}

\usepackage{enumitem}
\usepackage{mathtools}
\usepackage{steinmetz}
\usepackage{tikz}
\usepackage{circuitikz}
\usepackage{verbatim}
\usepackage{tfrupee}
\usepackage[breaklinks=true]{hyperref}
\usepackage{graphicx}
\usepackage{tkz-euclide}

\usetikzlibrary{calc,math}
\usepackage{listings}
    \usepackage{color}                                            %%
    \usepackage{array}                                            %%
    \usepackage{longtable}                                        %%
    \usepackage{calc}                                             %%
    \usepackage{multirow}                                         %%
    \usepackage{hhline}                                           %%
    \usepackage{ifthen}                                           %%
    \usepackage{lscape}     
\usepackage{multicol}
\usepackage{chngcntr}

\DeclareMathOperator*{\Res}{Res}

\renewcommand\thesection{\arabic{section}}
\renewcommand\thesubsection{\thesection.\arabic{subsection}}
\renewcommand\thesubsubsection{\thesubsection.\arabic{subsubsection}}

\renewcommand\thesectiondis{\arabic{section}}
\renewcommand\thesubsectiondis{\thesectiondis.\arabic{subsection}}
\renewcommand\thesubsubsectiondis{\thesubsectiondis.\arabic{subsubsection}}


\hyphenation{op-tical net-works semi-conduc-tor}
\def\inputGnumericTable{}                                 %%

\lstset{
%language=C,
frame=single, 
breaklines=true,
columns=fullflexible
}
\begin{document}


\newtheorem{theorem}{Theorem}[section]
\newtheorem{problem}{Problem}
\newtheorem{proposition}{Proposition}[section]
\newtheorem{lemma}{Lemma}[section]
\newtheorem{corollary}[theorem]{Corollary}
\newtheorem{example}{Example}[section]
\newtheorem{definition}[problem]{Definition}

\newcommand{\BEQA}{\begin{eqnarray}}
\newcommand{\EEQA}{\end{eqnarray}}
\newcommand{\define}{\stackrel{\triangle}{=}}
\bibliographystyle{IEEEtran}
\providecommand{\mbf}{\mathbf}
\providecommand{\pr}[1]{\ensuremath{\Pr\left(#1\right)}}
\providecommand{\qfunc}[1]{\ensuremath{Q\left(#1\right)}}
\providecommand{\sbrak}[1]{\ensuremath{{}\left[#1\right]}}
\providecommand{\lsbrak}[1]{\ensuremath{{}\left[#1\right.}}
\providecommand{\rsbrak}[1]{\ensuremath{{}\left.#1\right]}}
\providecommand{\brak}[1]{\ensuremath{\left(#1\right)}}
\providecommand{\lbrak}[1]{\ensuremath{\left(#1\right.}}
\providecommand{\rbrak}[1]{\ensuremath{\left.#1\right)}}
\providecommand{\cbrak}[1]{\ensuremath{\left\{#1\right\}}}
\providecommand{\lcbrak}[1]{\ensuremath{\left\{#1\right.}}
\providecommand{\rcbrak}[1]{\ensuremath{\left.#1\right\}}}
\theoremstyle{remark}
\newtheorem{rem}{Remark}
\newcommand{\sgn}{\mathop{\mathrm{sgn}}}
\providecommand{\abs}[1]{\left\vert#1\right\vert}
\providecommand{\res}[1]{\Res\displaylimits_{#1}} 
\providecommand{\norm}[1]{\left\lVert#1\right\rVert}
%\providecommand{\norm}[1]{\lVert#1\rVert}
\providecommand{\mtx}[1]{\mathbf{#1}}
\providecommand{\mean}[1]{E\left[ #1 \right]}
\providecommand{\fourier}{\overset{\mathcal{F}}{ \rightleftharpoons}}
%\providecommand{\hilbert}{\overset{\mathcal{H}}{ \rightleftharpoons}}
\providecommand{\system}{\overset{\mathcal{H}}{ \longleftrightarrow}}
	%\newcommand{\solution}[2]{\textbf{Solution:}{#1}}
\newcommand{\solution}{\noindent \textbf{Solution: }}
\newcommand{\cosec}{\,\text{cosec}\,}
\providecommand{\dec}[2]{\ensuremath{\overset{#1}{\underset{#2}{\gtrless}}}}
\newcommand{\myvec}[1]{\ensuremath{\begin{pmatrix}#1\end{pmatrix}}}
\newcommand{\mydet}[1]{\ensuremath{\begin{vmatrix}#1\end{vmatrix}}}
\numberwithin{equation}{subsection}
\makeatletter
\@addtoreset{figure}{problem}
\makeatother
\let\StandardTheFigure\thefigure
\let\vec\mathbf
\renewcommand{\thefigure}{\theproblem}
\def\putbox#1#2#3{\makebox[0in][l]{\makebox[#1][l]{}\raisebox{\baselineskip}[0in][0in]{\raisebox{#2}[0in][0in]{#3}}}}
     \def\rightbox#1{\makebox[0in][r]{#1}}
     \def\centbox#1{\makebox[0in]{#1}}
     \def\topbox#1{\raisebox{-\baselineskip}[0in][0in]{#1}}
     \def\midbox#1{\raisebox{-0.5\baselineskip}[0in][0in]{#1}}
\vspace{3cm}
\title{Assignment-7}
\author{Ankur Aditya - EE20RESCH11010}
\maketitle
\newpage
\bigskip
\renewcommand{\thefigure}{\theenumi}
\renewcommand{\thetable}{\theenumi}

\begin{abstract}
This document contains the procedure to find the equation of tangent to parabola.
\end{abstract}
Download the python code from 
\begin{lstlisting}
https://github.com/ankuraditya13/EE5609-Assignment7
\end{lstlisting}
%
and latex-file codes from 
%
\begin{lstlisting}
https://github.com/ankuraditya13/EE5609-Assignment7
\end{lstlisting}

\section{Problem}
Find the QR Decomposition of matrix,
\begin{align}
\vec{A} = \myvec{2&-6\\1&-2}
\label{Q}
\end{align}
\section{Solution}
Let $c_1$ and $c_2$ be the column vectors of given matrix $\vec{A}$
\begin{align}
c_1 = \myvec{2\\1}\\
c_2 = \myvec{-6\\-2}
\end{align}
We can express the matrix $\vec{A}$ as,
\begin{align}
\vec{A}=\vec{QR}
\end{align}
Where, $\vec{Q}$ is an orthogonal matrix given as,
\begin{align}
\vec{Q} = \myvec{\vec{u_1}&\vec{u_2}}\label{orthogonal}
\end{align}
and $\vec{R}$ is an upper triangular matrix given as,
\begin{align}
\vec{R} = \myvec{k_1&r_1\\0&k_2}
\label{UT}
\end{align}
Now, we can express $\alpha$ and $\beta$ as,
\begin{align}
c_1 = k_1\vec{u_1}\label{alpha}\\
c_2 = r_1\vec{u_1} +k_2\vec{u_2}\label{beta}\\
\mbox{where, } k_1 = \norm{c_1} = \sqrt{2^2+(1^2} = \sqrt{5}\label{k1}
\end{align} 
Solving equation \eqref{alpha} for $\vec{u_1}$ ,
\begin{align}
\vec{u_1} = \frac{c_1}{k_1} = \frac{1}{\sqrt{5}}\myvec{2\\1}\label{u1}
\end{align}
\begin{align}
\mbox{Now, } r_1 = \frac{\vec{u_1}^Tc_2}{\norm{\vec{u_1}}^2}\\
\implies \frac{\frac{1}{\sqrt{5}}\myvec{2&1}\myvec{-6\\-2}}{1}
\end{align}
\begin{align}
\mbox{Hence, } r_1 = -\frac{-14}{\sqrt{5}}
\label{r1}\\
\vec{u_2} = \frac{c_2-r_1\vec{u_1}}{\norm{c_2-r_1\vec{u_1}}}\\
\implies \frac{\myvec{-6\\-2} - \brak{\frac{-14}{\sqrt{5}}}\brak{\frac{1}{\sqrt{5}}\myvec{2\\1}}}{\norm{\myvec{-6\\-2} - \brak{-\frac{14}{\sqrt{5}}\frac{1}{\sqrt{5}}\myvec{2\\1}}}} 
\end{align}
\begin{align}
\implies\vec{u_2} = \frac{1}{\sqrt{5}}\myvec{-1\\2}\label{u2}
\end{align}
\begin{align}
\mbox{Now, } k_2 = u_2^Tc_2\\
\implies \frac{1}{\sqrt{5}}\myvec{-1&2}\myvec{-6\\-2}\\\implies k_2 = \frac{2}{\sqrt{5}} \label{k2}
\end{align}
Hence substituting the values of unknown parameter from equations \eqref{k1}, \eqref{k2}, \eqref{u1}, \eqref{u2} and \eqref{r1} to equation \eqref{orthogonal} and \eqref{UT} we get,
\begin{align}
\vec{Q} = \frac{1}{\sqrt{5}}\myvec{2&-1\\1&2}
\end{align} 
\begin{align}
\vec{R} = \myvec{\sqrt{5}&\frac{-14}{\sqrt{5}}\\0&\frac{2}{\sqrt{5}}}
\end{align}
\end{document}